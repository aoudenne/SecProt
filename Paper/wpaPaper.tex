

\documentclass[11pt, twocolumn]{article} % use larger type; default would be 10pt

\usepackage[utf8]{inputenc} % set input encoding (not needed with XeLaTeX)
\usepackage{amsmath}
\usepackage{enumitem}

%%% PAGE DIMENSIONS
%\usepackage{geometry} % to change the page dimensions
%\geometry{letterpaper} % or letterpaper (US) or a5paper or....


\usepackage{graphicx} % support the \includegraphics command and options

% \usepackage[parfill]{parskip} % Activate to begin paragraphs with an empty line rather than an indent


\usepackage{verbatim} % adds environment for commenting out blocks of text & for better verbatim
\usepackage{moreverb} %for tabs in verbatim env
\usepackage{subfig} % make it possible to include more than one captioned figure/table in a single float


%%% HEADERS & FOOTERS
%\usepackage{fancyhdr} % This should be set AFTER setting up the page geometry
%\pagestyle{fancy} % options: empty , plain , fancy
%\renewcommand{\headrulewidth}{0pt} % customise the layout...
%\lhead{}\chead{}\rhead{}
%\lfoot{}\cfoot{\thepage}\rfoot{}



\title{\textbf{Brief yet Scintillating Article about WEP and WPA2}}
\author{
	Ashley Oudenne\\
	The University of Texas at Austin\\
	aoudenne@cs.utexas.edu\\
	\and
	Jerremy Adams\\
	The University of Texas at Austin\\
	ja7872@cs.utexas.edu
}
%\date{} % Activate to display a given date or no date (if empty),
         % otherwise the current date is printed 

\begin{document}
\maketitle
\begin{abstract}
Interesting vague things about our paper.
200 words or less, and should say something like:
In this paper, we present the WEP and WPA Four-Way Handshake protocols in detail and formally model them using the ProVerif Cryptographic Verifier. 
\end{abstract}

\section{Introduction}
Wireless communication is a staple of 21st century daily life, and security is a serious concern for individuals, businesses, and organizations looking to communicate quickly and securely.  To ensure this, different standards have been proposed that define protocols for secure communication.  

One such standard was the 802.11 standard ratified in September of 1999, which intended to provide data confidentiality through Wired Equivalent Privacy, or WEP \cite{IEEE802.11}.  WEP uses a 40- or 104-bit encryption key that is manually entered into access points and devices.  This key never changes, so if it is compromised, all future messages on the network are compromised until every device is manually rekeyed.  The intention of WEP was to provide the same level of confidentiality as that of a traditional wired network.  

Unfortunately, WEP relied on the RC4 stream cipher, which was thought to be secure but which was actually vulnerable to attacks.  The Cyclic Redundancy Check (CRC) checksum algorithm used by WEP does not provide a strong enough integrity guarantee, because it permits the guessing of individual bytes of a packet \cite{bulbul08}.  Since the CRC is simple a linear function of the message, an attacker can modify an encrypted message and fix the checksum so that the message appears not to have been modified.

To remedy this, IEEE released WPA in 2003 as a temporary remedy for WEP until a new standard could be ratified.  WPA was designed to work on devices that were currently using WEP until new hardware was created, so it also uses the RC4 cipher and the CRC checksum mechanism \cite{wpa03}.  However, it uses the Temporal Key Integrity Protocol, which implements a key mixing function that combines the root key with an initialization vector \textit{before} passing it to RC4, instead of just concatenating it as in WEP.  This prevents related key attacks, to which WEP is susceptible.  TKIP also enforces rekeying and sequence counters to thwart replay attacks.  It also includes an additional message integrity check called Michael that increases security.  Although WPA provides greater security than WEP, it is still susceptible to some of the same attacks as WEP, because of the insecurity of the RC4 cipher and the CRC checksum algorithm.  As a result, the 802.11i standard was ratified in June of 2004 to replace the use of the TKIP protocol (which uses RC4) with AES-based CCMP encryption\cite{IEEE802.11i}. This provides strong security for key generation.

Both WPA and WPA2 rely on the computation of a secure key to encrypt data.  To avoid the insecurity of WEP (in which the means of calculating the key are sent over the network), both parties begin with the same Pairwise Master Key (PMK) and use this key to compute the Pairwise Transient Key (PTK) which is actually used to encrypt data.  Neither of these keys are ever sent over the network.  Instead, the wireless access point and the supplicant device engage in the Four-Way Handshake protocol in order to transmit the data necessary to calculate the PTK.  

While the Four-Way Handshake is immune to most attacks, it is vulnerable to a Denial-of-Service (DOS) attack.  This attack prevents the station and the access point from ever fully authenticating with one another.  While an attack of this nature is not particularly devastating, in that it does not leak keys or permit the attacker to corrupt data, the availability of a network should not be influenced by an attacker.

In this paper, we present the WEP and WPA Four-Way Handshake protocols in detail and formally model them using the ProVerif Cryptographic Verifier. In Section \ref{sec:Related Work}, we present work by other researchers on the security of WEP and 802.11i.  We particularly focus on the security of the Four-Way Handshake protocol.  We explain the protocols and how we model them in Section \ref{sec:model}.  We also describe ProVerif and our attacker model.  In Section \ref{sec:analysis}, we show the possible attacks on the WEP protocol and what particular aspects of the protocol lead to these attacks.  We will then explain how these aspects have been eliminated in WPA/WPA2 due to the Four-Way Handshake Protocol, and then demonstrate how a Denial-Of-Service attack is still possible.  Finally, in Section \ref{sec:conclude}, we summarize our conclusions and suggest future work that could be done on proving the correctness of wireless security.
\section{Related Work}
\label{sec:Related Work}
There has been much work, both with automatic verifiers and without, on proving the insecurity of WEP.  In \cite{borisov01}, a large number of insecurities in WEP are discussed without the aid of an automatic verifier.  The authors highlight possible attacks resulting from the risk of keystream reuse, due to the fact that encrypting two messages under the same keystream can reveal information about both messages.  Since the initialization vectors used to compute the keystreams are re-initialized every time a wireless card is re-inserted into a device, there are many opportunities for an attack of this nature.  

The authors also discuss the risk of message modification in WEP due to the CRC checksum being a linear function of the message, which means that it distributes over XOR.  An attacker can arbitrarily modify even messages he hasn't decrypted by simply XORing the ciphertext with some bitstream and the checksum of the bitstream.  Messages can be injected into a network because CRC is not dependent on the keystream used to encode a message. Once an attacker learns an initialization vector and its corresponding keystream, the keystream can be reused indefinitely to insert new messages into the network, because initialization vectors are never checked for freshness.  This also allows an attacker to authenticate himself to the network.  

The security of 802.11i with respect to confidentialty and authentication has also been claimed without the use of automatic verifiers.  In \cite{he05}, He and Mitchell consider each stage of the protocol and argue its security from a number of possible threats, including malicious access points, session hijacking, eavesdropping, and message deletion.  They conclude that it provides effective confidentiality and integrity when the CCMP protocol is used, and that it may provide satisfactory mutual authentication and key management.  However, they identify several possible Denial of Service attacks, since the protocol is not designed to ensure liveness.

One of these Denial of Service attacks, identified in \cite{he04} and \cite{liu08}, deals with the Four-Way Handshake protocol that is responsible for establishing the Pairwise Transient Key (PTK).  Using automatic verification tools, both groups of researchers prove that it is possible to launch a Denial of Service attack on a supplicant in which the supplicant is continually forced to regenerate a new but incorrect PTK, preventing it from ever communicating with the server.  \cite{he04} solves this problem by suggesting the reuse of the supplicant's nonce until after the PTK has been established.  However, this solution is susceptible to a Denial of Service attack against CPU resources.  \cite{liu08} proposes a different solution intended to avoid this.  They suggest the addition of a large random number nonce that is sent along with traditional participant's nonce.  However, both of these nonces are encrypted with the Pairwise Master Key, which is assumed to be unknown to the attacker.  This ensures that attackers cannot flood the network with nonces in an attempt to trick the supplicant into continuously recomputing the PTK.

Automatic verification has also been applied to WEP.  Lafourcade et al. found an attack similar to \cite{borisov01} described above using the ProVerif Cryptographic Verifier. By placing multiple messages on the channel encrypted with the same keystream, the attacker is able to recover the contents of encrypted messages.  To prevent this attack, they then implemented a version of WEP in which all intialization vectors were guaranteed to be unique.  This protocol was considered secure by ProVerif.  Unfortunately it is impossible to ensure that initialization vectors will always be unique in the real world, which is why WEP is no longer considered to be secure.
 
\section{Modeling of Protocols}
\label{sec:model}
\subsection{Protocol Descriptions}
\subsubsection{WEP}
WEP, described in \cite{borisov01}, uses a one-message protocol to transmit data between two parties that relies on a secret key \textit{k} that has previously been shared between them.  The intention of this protocol is to provide the same confidentiality as that of a wired network \cite{IEEE802.11}.  Before a message is sent, an integrity checksum \textit{C(m)} is computed on the message so that the recipient can verify that the message has not been altered in transit.  The message is concatenated to this checksum to form the plaintext \textit{P}.

The plaintext is now encrypted using the RC4 cipher.  The sender chooses an initialization vector IV and uses this vector along with the secret key \textit{k} to generate an arbitrary length sequence of pseudorandom bits known as a keystream \cite{spore}.  The sender then uses exclusive-or (or XOR, denoted by $\oplus$) to XOR \textit{P} and the keystream to generate ciphertext \textit{C}.  The complete encryption process can be represented by: 
$$C = P \oplus RC4(IV, k)$$
The message is then ready to be transmitted from the sender to the receiver.  We will represent this transmission symbolically as:
$$A \rightarrow B: IV, (P \oplus RC4(IV, k)),$$
$$ \text{where} P = \langle m, C(m) \rangle$$ 

Notice that the initialization vector \textit{IV} is sent in the clear, meaning that anyone can read the value of the initialization vector.  This should not matter because an attacker would need boh the \textit{IV} and the secret key \textit{k} to recover the keystream used to decrypt the message, but we will show later in this paper that this is enough to break WEP.

Decryption works by reversing the encryption process described above.  The recipient first regenerates the keystream used to encode the message with the secret key \textit{k} and the \textit{IV} sent along with the encrypted message.  He can XOR the ciphertext with the keystream to recover the plaintext \textit{P}:
\begin{align*}
P &= C \oplus RC4(IV, k)\\
\nonumber &= (P \oplus RC4(IV, k)) \oplus RC4(IV, k)\\
\nonumber &= P
\end{align*}
The recipient can then verify the integrity of the message by splitting \textit{P} into $\langle m, C(m) \rangle$ and recomputing the checksum of \textit{m}.  If the computed checksum matches the sent checksum, then it was incorrectly assumed that message has not been tampered with.   However, as we discuss in Section \ref{sec:Related Work}, the checksum is not enough to provide integrity. 

{%We should say something like: The protocol intends for this to guarantee integrity, except not shitty.   quote: The integrity check field is implemented as a CRC-32 checksum, which is part of the encrypted payload of the packet. However, CRC-32 is linear, which means that it is possible to compute the bit difference of two CRCs based on the bit difference of the messages over which they are taken. In other words, flipping bit n in the message results in a deterministic set of bits in the CRC that must be flipped to produce a correct checksum on the modified message. Because flipping bits carries through after an RC4 decryption, this allows the attacker to flip arbitrary bits in an encrypted message and correctly adjust the checksum so that the resulting message appears valid.%}

\subsubsection{WPA/WPA2 Four-Way Handshake}
The Four-Way Handshake protocol is used in both WPA and WPA2 to authenticate a station to an access point, to compute a pairwise transient key (PTK) to be used in future communication between these parties, and to distribute a group transient key (GTK) to be used by the station to communicate with other devices connected to the access point \cite{liu08}.  Once generated, the PTK is broken up into five different keys, but for the purposes of modeling the protocol it is sufficient to think of it as a single key.  We can assume that both the station and the access point begin by knowing the Pairwise Master Key (PMK), which will be used to compute the PTK.  The PMK is either computed by both parties in enterprise mode, or known ahead of time in pre-shared key mode (used for personal networks).

The Four-Way Handshake proceeds as follows.  An access point and a wireless station each compute their own fresh random nonce, $N_A$ and $N_S$, respectively.  To begin the protocol, an access point sends a wireless station $N_A$ in the clear with no guarantee of integrity.  Once the station receives this nonce, it actually has all the information it needs to construct the PTK.  It does so by concatenating five values: the PMK, the access point's nonce, it's own fresh random nonce, the MAC address of the access point, and the station's own MAC address.  This concatenated value is then passed through a cryptographic hash function to derive the PTK.  

Next, the station sends its nonce back to the access point along with a Message Authentication and Integrity code (MAIC) created by running a MAC algorithm on the nonce using the PTK as the secret key.  Once the access point has received the station's nonce and the corresponding MAIC, it can construct the PTK for itself.  Once it has the PTK, the access point can run the same MAC algorithm on the station's nonce to ensure both parties have a consistent PTK.  Notice that if the first message (the access point's nonce) was tampered with, the station and access point would not have a consistent PTK, and thus the integrity of the first message is guaranteed here.  At this point, the access point can reason that the station is legitimate as its PTK was derived from the shared secret PMK.  

To authenticate the access point to the station, the access point sends the GTK encrypted by the PTK along with another MAIC, also derived from the PTK.  Upon receipt of this message by the station, the station can reason about the identity of the access point as above.  That is, it can calculate the MAIC for the encrypted GTK and see that the access point is legitimate since it has constructed the PTK from the shared secret PMK.  

Finally, the station merely sends back an acknowledgement to the access point, and both the access point and station install the keys for their communication session.  

This protocol can be summarized symbolically as: 

\begin{enumerate}[leftmargin=5mm]
{\small
\item $AP \rightarrow STA: N_A$

{\tiny (STA calculates PTK)}

\item $STA \rightarrow AP: N_S, MAIC(N_S)$

{\tiny (AP calculates PTK)

(AP authenticates STA by verifying $MAIC(N_S)$) }

\item $AP \rightarrow STA: \{GTK\}_{PTK}, MAIC(\{GTK\}_{PTK})$

{\tiny (STA authenticates AP by verifying $MAIC(\{GTK\}_{PTK})$) }

\item $STA \rightarrow AP: ACK$

{\tiny (STA, AP install PTK for use in this session) }
}
\end{enumerate}


\subsection{ProVerif}
To formally prove the correctness of WEP and the Four-Way Handshake, we use the ProVerif Cryptographic Verifier created by Bruno Blanchet.  ProVerif uses prolog rules to encode the protocol and abstracts away fresh values and the number of steps in the protocol \cite{blanchet01}.  Instead, Proverif treats each fresh value as a function of other messages in the protocol, meaning that different values are used for each pair of protocol participants.  Each step in the protocol can be completed any number of times, and past steps can be re-executed arbitrarily.  This permits ProVerif to execute an unlimited number of runs of a protocol.  A protocol is proved to be secure with respect to some invariant by querying  whether ProVerif can generate the inverse of that invariant.  For example, a confidentiality invariant is proven by querying whether the attacker can learn a secret (e.g. a private key, or the contents of an encrypted message). Since we assume a Dolev-Yao attacker model, we assume that all messages on a channel \textit{c} are owned by the attacker.  The attacker can read, modify, generate, or delete any message he desires (though he cannot learn the contents of an encrypted message without the corresponding key). Since the attacker has unlimited access to all messages on \textit{c}, an invariant is proved if the attacker cannot invert that invariant despite access to all the messages. This verifier has been used to successfully prove the insecurity of protocols such as the Diffe-Hellman key exchange protocol, Initial Key Agreement, and the Needham-Schroeder symmetric-key protocol \cite{lafourcade10, abadi}.

\subsubsection{Horn Clauses}
ProVerif can take protocols in either horn clauses or pi-calculus.  In our protocol implementations, we chose to use typed horn clauses.  An untyped horn clause is a disjunction of literals with at most one positive literal (ex. $ \neg p \lor \neg q \lor t$) \cite{blanchet09}.  They can be written as implications (ex. $(p \land q ) \rightarrow t$), as they are in Prolog, on which ProVerif is based. Typed horn clauses merely allow the user to add a type system to the program for clarity and convenience, but the underlying logic resolution algorithm is the same as that of untyped horn clauses.

\subsubsection{Attacker Model}
In ProVerif, it is not necessary to explicitly model the attacker.  Rather, the user constructs a list of clauses detailing what anyone, including the attacker, can do with messages that are put on the channel \textit{c}.  Such abilities include separating a message $c(a,b)$ into its component parts and placing those parts on the channel ($(c(a)$, $c(b)$), decrypting messages if the encryption key of the message is known, encrypting messages using a known key, XOR-ing messages, and computing checksums of messages.  We use the Dolev-Yao attacker model, in which the attacker can intercept, overhear, and create new messages, because it a very strong model.  Since the attacker has complete control of the message, protecting a wireless protocol against a Dolev-Yao attacker should provide realistic security guarantees for real-world attackers. 

\subsection{Modeling Security Properties}

\section{Analysis of Protocol Models}
\label{sec:analysis}
\subsection{WEP}
In order to implement WEP in ProVerif, we first needed to identify an invariant that must hold true if the protocol is to be considered secure.  Since an attacker can use XOR to modify messages and decrypt messages encrypted using the same keystream, we chose as our invariant the property that if an attacker XORs two different messages on the channel that were encrypted using the same keystream, then a secret has been learned.  In ProVerif, this invariant can be modeled using ``forall" clauses on the possible principals in the protocol:

 %the 4 is the number of tabbed spaces
\begin{verbatimtab}[4] 
forall p:principal, q:principal; 
	c(xor((m[p],checkSum(m[p])),
		(m[q],checkSum(m[q])))) 
			& p <> q -> c(secret[]).
\end{verbatimtab}


\subsection{WPA/WPA2 Four-Way Handshake}

\section{Conclusions and Future Work}
\label{sec:conclude}
\appendix
\section{ProVerif Implementation of WEP}
\section{ProVerif Implementation of Four-Way Handshake}
\bibliographystyle{plain}
\bibliography{wpaPaper}
\end{document}