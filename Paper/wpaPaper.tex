

\documentclass[11pt, twocolumn]{article} % use larger type; default would be 10pt

\usepackage[utf8]{inputenc} % set input encoding (not needed with XeLaTeX)
\usepackage{amsmath}

%%% PAGE DIMENSIONS
%\usepackage{geometry} % to change the page dimensions
%\geometry{letterpaper} % or letterpaper (US) or a5paper or....


\usepackage{graphicx} % support the \includegraphics command and options

% \usepackage[parfill]{parskip} % Activate to begin paragraphs with an empty line rather than an indent


\usepackage{verbatim} % adds environment for commenting out blocks of text & for better verbatim
\usepackage{subfig} % make it possible to include more than one captioned figure/table in a single float


%%% HEADERS & FOOTERS
%\usepackage{fancyhdr} % This should be set AFTER setting up the page geometry
%\pagestyle{fancy} % options: empty , plain , fancy
%\renewcommand{\headrulewidth}{0pt} % customise the layout...
%\lhead{}\chead{}\rhead{}
%\lfoot{}\cfoot{\thepage}\rfoot{}



\title{\textbf{Brief yet Scintillating Article about WEP and WPA2}}
\author{
	Ashley Oudenne\\
	The University of Texas at Austin\\
	aoudenne@cs.utexas.edu\\
	\and
	Jerremy Adams\\
	The University of Texas at Austin\\
	ja7872@cs.utexas.edu
}
%\date{} % Activate to display a given date or no date (if empty),
         % otherwise the current date is printed 

\begin{document}
\maketitle
\begin{abstract}
Interesting yet vague things about our paper.
\end{abstract}

\section{Introduction}

\section{Related Work}

\section{Modeling of Protocols}
\subsection{Protocol Structures}
\subsubsection{WEP}
WEP, described in \cite{borisov01}, uses a one-message protocol to transmit data between two parties that relies on a secret key \textit{k} that has been shared between them.  Before a message is sent, an integrity checksum \textit{C(m)} is computed on the message so that the recipient can verify that the message has not been altered in transit.  The message is concatenated to this checksum to form the plaintext \textit{P}.

The plaintext is now encrypted using the RC4 cipher.  The sender chooses an initialization vector IV and uses this vector along with the secret key \textit{k} to generate a keystream, which is a long sequence of pseudorandom bits.  The sender then uses exclusive-or (or XOR, denoted by $\oplus$) to XOR \textit{P} and the keystream to generate ciphertext \textit{C}.  The complete encryption process can be represented by: 
$$C = P \oplus RC4(IV, k)$$

The message is then ready to be transmitted from the sender to the receiver.  We will represent this transmission symbolically as:
$$A \rightarrow B: IV, (P \oplus RC4(IV, k)), where P = \langle m, C(m) \rangle$$ 

Notice that the initialization vector \textit{IV} is sent in the clear, meaning that anyone can read the value of the initialization vector.  This should not matter because an attacker would need boh the \textit{IV} and the secret key \textit{k} to recover the keystream used to decrypt the message, but we will show later in this paper that this is enough to break WEP.

Decryption works by reversing the encryption process described above.  The recipient first regenerates the keystream used to encode the message with the secret key \textit{k} and the \textit{IV} sent along with the encrypted message.  He can XOR the ciphertext with the keystream to recover the plaintext \textit{P}:
\begin{align*}
\label{}
P &= C \oplus RC4(IV, k)\\
\nonumber &= (P \oplus RC4(IV, k)) \oplus RC4(IV, k)\\
\nonumber &= P
\end{align*}
The recipient can then verify the integrity of the message by splitting \textit{P} into $\langle m, C(m) \rangle$ and recomputing the checksum of \textit{m}.  If the computed checksum matches the sent checksum, then the message has not been tampered with.  

\subsubsection{WPA/WPA2 Four-Way Handshake}
The Four-Way Handshake protocol is used in both WPA and WPA2 to authenticate a station to an access point, to compute a pairwise transient key (PTK) to be used in future communication between these parties, and to distribute a group transient key (GTK) to be used by the station to communicate with other devices connected to the access point \cite{liu08}.  
\subsection{ProVerif}
\subsection{Attacker Model}
\subsection{Modeling Security Properties}

\section{Analysis of Protocol Models}
\subsection{WEP}
\subsection{WPA/WPA2 Four-Way Handshake}

\section{Conclusions and Future Work}

\bibliographystyle{plain}
\bibliography{wpaPaper}
\end{document}